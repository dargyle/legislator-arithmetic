\documentclass[11pt,]{article}
\usepackage[left=1in,top=1in,right=1in,bottom=1in]{geometry}
\newcommand*{\authorfont}{\fontfamily{phv}\selectfont}
\usepackage[]{mathpazo}


  \usepackage[T1]{fontenc}
  \usepackage[utf8]{inputenc}



\usepackage{abstract}
\renewcommand{\abstractname}{}    % clear the title
\renewcommand{\absnamepos}{empty} % originally center

\renewenvironment{abstract}
 {{%
    \setlength{\leftmargin}{0mm}
    \setlength{\rightmargin}{\leftmargin}%
  }%
  \relax}
 {\endlist}

\makeatletter
\def\@maketitle{%
  \newpage
%  \null
%  \vskip 2em%
%  \begin{center}%
  \let \footnote \thanks
    {\fontsize{18}{20}\selectfont\raggedright  \setlength{\parindent}{0pt} \@title \par}%
}
%\fi
\makeatother




\setcounter{secnumdepth}{0}

\usepackage{color}
\usepackage{fancyvrb}
\newcommand{\VerbBar}{|}
\newcommand{\VERB}{\Verb[commandchars=\\\{\}]}
\DefineVerbatimEnvironment{Highlighting}{Verbatim}{commandchars=\\\{\}}
% Add ',fontsize=\small' for more characters per line
\usepackage{framed}
\definecolor{shadecolor}{RGB}{248,248,248}
\newenvironment{Shaded}{\begin{snugshade}}{\end{snugshade}}
\newcommand{\KeywordTok}[1]{\textcolor[rgb]{0.13,0.29,0.53}{\textbf{#1}}}
\newcommand{\DataTypeTok}[1]{\textcolor[rgb]{0.13,0.29,0.53}{#1}}
\newcommand{\DecValTok}[1]{\textcolor[rgb]{0.00,0.00,0.81}{#1}}
\newcommand{\BaseNTok}[1]{\textcolor[rgb]{0.00,0.00,0.81}{#1}}
\newcommand{\FloatTok}[1]{\textcolor[rgb]{0.00,0.00,0.81}{#1}}
\newcommand{\ConstantTok}[1]{\textcolor[rgb]{0.00,0.00,0.00}{#1}}
\newcommand{\CharTok}[1]{\textcolor[rgb]{0.31,0.60,0.02}{#1}}
\newcommand{\SpecialCharTok}[1]{\textcolor[rgb]{0.00,0.00,0.00}{#1}}
\newcommand{\StringTok}[1]{\textcolor[rgb]{0.31,0.60,0.02}{#1}}
\newcommand{\VerbatimStringTok}[1]{\textcolor[rgb]{0.31,0.60,0.02}{#1}}
\newcommand{\SpecialStringTok}[1]{\textcolor[rgb]{0.31,0.60,0.02}{#1}}
\newcommand{\ImportTok}[1]{#1}
\newcommand{\CommentTok}[1]{\textcolor[rgb]{0.56,0.35,0.01}{\textit{#1}}}
\newcommand{\DocumentationTok}[1]{\textcolor[rgb]{0.56,0.35,0.01}{\textbf{\textit{#1}}}}
\newcommand{\AnnotationTok}[1]{\textcolor[rgb]{0.56,0.35,0.01}{\textbf{\textit{#1}}}}
\newcommand{\CommentVarTok}[1]{\textcolor[rgb]{0.56,0.35,0.01}{\textbf{\textit{#1}}}}
\newcommand{\OtherTok}[1]{\textcolor[rgb]{0.56,0.35,0.01}{#1}}
\newcommand{\FunctionTok}[1]{\textcolor[rgb]{0.00,0.00,0.00}{#1}}
\newcommand{\VariableTok}[1]{\textcolor[rgb]{0.00,0.00,0.00}{#1}}
\newcommand{\ControlFlowTok}[1]{\textcolor[rgb]{0.13,0.29,0.53}{\textbf{#1}}}
\newcommand{\OperatorTok}[1]{\textcolor[rgb]{0.81,0.36,0.00}{\textbf{#1}}}
\newcommand{\BuiltInTok}[1]{#1}
\newcommand{\ExtensionTok}[1]{#1}
\newcommand{\PreprocessorTok}[1]{\textcolor[rgb]{0.56,0.35,0.01}{\textit{#1}}}
\newcommand{\AttributeTok}[1]{\textcolor[rgb]{0.77,0.63,0.00}{#1}}
\newcommand{\RegionMarkerTok}[1]{#1}
\newcommand{\InformationTok}[1]{\textcolor[rgb]{0.56,0.35,0.01}{\textbf{\textit{#1}}}}
\newcommand{\WarningTok}[1]{\textcolor[rgb]{0.56,0.35,0.01}{\textbf{\textit{#1}}}}
\newcommand{\AlertTok}[1]{\textcolor[rgb]{0.94,0.16,0.16}{#1}}
\newcommand{\ErrorTok}[1]{\textcolor[rgb]{0.64,0.00,0.00}{\textbf{#1}}}
\newcommand{\NormalTok}[1]{#1}


\title{Legislator Arithmetic \thanks{The code for this method is available at the author's github repository.}  }



\author{\Large Daniel Argyle\vspace{0.05in} \newline\normalsize\emph{FiscalNote}  }


\date{}

\usepackage{titlesec}

\titleformat*{\section}{\normalsize\bfseries}
\titleformat*{\subsection}{\normalsize\itshape}
\titleformat*{\subsubsection}{\normalsize\itshape}
\titleformat*{\paragraph}{\normalsize\itshape}
\titleformat*{\subparagraph}{\normalsize\itshape}


\usepackage{natbib}
\bibliographystyle{plainnat}
\usepackage[strings]{underscore} % protect underscores in most circumstances



\newtheorem{hypothesis}{Hypothesis}
\usepackage{setspace}

\makeatletter
\@ifpackageloaded{hyperref}{}{%
\ifxetex
  \PassOptionsToPackage{hyphens}{url}\usepackage[setpagesize=false, % page size defined by xetex
              unicode=false, % unicode breaks when used with xetex
              xetex]{hyperref}
\else
  \PassOptionsToPackage{hyphens}{url}\usepackage[unicode=true]{hyperref}
\fi
}

\@ifpackageloaded{color}{
    \PassOptionsToPackage{usenames,dvipsnames}{color}
}{%
    \usepackage[usenames,dvipsnames]{color}
}
\makeatother
\hypersetup{breaklinks=true,
            bookmarks=true,
            pdfauthor={Daniel Argyle (FiscalNote)},
             pdfkeywords = {ideal point estimation},  
            pdftitle={Legislator Arithmetic},
            colorlinks=true,
            citecolor=blue,
            urlcolor=blue,
            linkcolor=magenta,
            pdfborder={0 0 0}}
\urlstyle{same}  % don't use monospace font for urls

% set default figure placement to htbp
\makeatletter
\def\fps@figure{htbp}
\makeatother



% add tightlist ----------
\providecommand{\tightlist}{%
\setlength{\itemsep}{0pt}\setlength{\parskip}{0pt}}

\begin{document}
	
% \pagenumbering{arabic}% resets `page` counter to 1 
%
% \maketitle

{% \usefont{T1}{pnc}{m}{n}
\setlength{\parindent}{0pt}
\thispagestyle{plain}
{\fontsize{18}{20}\selectfont\raggedright 
\maketitle  % title \par  

}

{
   \vskip 13.5pt\relax \normalsize\fontsize{11}{12} 
\textbf{\authorfont Daniel Argyle} \hskip 15pt \emph{\small FiscalNote}   

}

}








\begin{abstract}

    \hbox{\vrule height .2pt width 39.14pc}

    \vskip 8.5pt % \small 

\noindent See intro\ldots{}


\vskip 8.5pt \noindent \emph{Keywords}: ideal point estimation \par

    \hbox{\vrule height .2pt width 39.14pc}



\end{abstract}


\vskip 6.5pt


\noindent  \section{Introduction}\label{introduction}

We propose a neural network implementation of ideal-point estimation
that scales well to large datasets and allows incorporation of
additional metadata. Neural networks are well-suited for these models,
and the performance benefit, along with distributed computing
capabilities, allows application of ideal point estimation to pooled
datasets where computation was previously infeasible due to scale. We
demonstrate the algorithm on two different datasets, the complete
history of US Congressional roll call votes and modern cosponsorship
networks, and compare the results against standard ideal point
estimation techniques.

To evaluate algorithmic performance, we test the resulting estimates on
both training and test data by holding out a subset of legislators'
votes. This allows us to compare the quality of different model
parameterizations and choice of dimensions while still guarding against
overfitting. Specifically, we directly compare the performance of
different ideal point parameterizations such as DW-NOMINATE and the
conventional Bayesian parameterization.

We demonstrate the algorithms in two ways. First, we jointly estimate
ideal points over the pooled set of US Congressional roll call votes
from 1789-2018. Unidimensional ideal points from the neural network
implementation are similar to the conventional DW-NOMINATE results.
However, cross validation scores indicate that the data are better
explained with more than one dimension. Clustering the multidimensional
ideal points yields intuitive temporal and ideological groupings and
provides a more nuanced picture of ideological polarization.

Second, we take advantage of the fact that many more bills are sponsored
than actually come to a vote and estimate an ideal point distribution
over a large set of sponsorship and cosponsorship decisions in the
93rd-114th Congresses. Cosponsorship provides a different perspective on
legislators' beliefs, independent of strategic voting or administrative
votes of little ideological salience. We treat cosponsorship as a clear
endorsement of a bill's content and assume that a choice not to
cosponsor a bill can be interpreted as something less than full support.
When compared to traditional ideal points, cosponsorship ideal points
show somewhat different trends in polarization and result in a higher
number of optimal dimensions.

\section{Existing methods}\label{existing-methods}

{[}This is polmeth, y'all know this already{]}

This work is inspired by two, highly similar, lines of research in
political science and computer science. Speaking generally\footnote{A
  complete review of the literature in either field is beyond the scope
  of this work.{]}} , political scientists--from the days of (really old
cite), through Poole and Rosenthal, and up to and including modern
contributions such as (modern cites)--have focused primarily on ideal
point estimation as a means to study the ideology space implied by the
ideal points themselves. That these methods also predict votes is
somewhat of an afterthought. On the other hand, computer science
implementations largely focus on predicting legislator votes, without
concern regarding the ideal points themselves.

\section{Technical Details}\label{technical-details}

\section{Results}\label{results}

\subsection{My method and WNOMINATE packages are
similar}\label{my-method-and-wnominate-packages-are-similar}

\subsection{Let's do some things that weren't really feasible
before}\label{lets-do-some-things-that-werent-really-feasible-before}

\begin{enumerate}
\def\labelenumi{\arabic{enumi}.}
\tightlist
\item
  I implemented DW-NOMINATE as a neural network

  \begin{itemize}
  \tightlist
  \item
    Why? Because I could! But also because it's a nice platform for this
    kind of optimization.
  \item
    It scales much better than existing implementations
  \item
    It's extensible in very interesting ways
  \end{itemize}
\item
  All ideal point models are (a bit) overfit.

  \begin{itemize}
  \tightlist
  \item
    At some point the algorithm starts to make marginal improvements to
    the parameters that don't improve out of sample performance
  \item
    Out of sample performance matters much more than in sample (e.g.~if
    we only cared about in sample we'd just add dimensions until we cam
    predict it perfectly)
  \item
    Out of sample performance is a useful metric for evaluating modeling
    choices (adding another dimension, adding a time component, adding
    an external variable)
  \end{itemize}
\end{enumerate}

this is inline 42 and

\begin{Shaded}
\begin{Highlighting}[]
\NormalTok{py}\OperatorTok{$}\NormalTok{answer}
\end{Highlighting}
\end{Shaded}

\begin{verbatim}
## [1] "42"
\end{verbatim}




\newpage
\singlespacing 
\end{document}
